% Modified based on Xiaoming Sun's template and https://www.overleaf.com/latex/templates/cs6780-assignment-template/hgjyygbykvrf

\documentclass[a4 paper,12pt]{article}
\usepackage[inner=2.0cm,outer=2.0cm,top=2.0cm,bottom=2.0cm]{geometry}
\linespread{1.1}
\usepackage{setspace}
\usepackage[rgb]{xcolor}
\usepackage{verbatim}
\usepackage{subcaption}
\usepackage{fancyhdr}
\usepackage{fullpage}
\usepackage[colorlinks=true, urlcolor=blue, linkcolor=blue, citecolor=blue]{hyperref}
\usepackage{booktabs}
\usepackage{amsmath,amsfonts,amsthm,amssymb}
\usepackage[shortlabels]{enumitem}
\usepackage{setspace}
\usepackage{extramarks}
\usepackage{soul,color}
\usepackage{graphicx,float,wrapfig}
\newcommand{\homework}[3]{
	\pagestyle{myheadings}
	\thispagestyle{plain}
	\newpage
	\setcounter{page}{1}
	\noindent
	\begin{center}
		\framebox{
			\vbox{\vspace{2mm}
				\hbox to 6.28in { {\bf Deep Learning \hfill} {\hfill {\rm #2} {\rm #3}} }
				\vspace{4mm}
				\hbox to 6.28in { {\Large \hfill #1  \hfill} }
				\vspace{3mm}}
		}
	\end{center}
	\vspace*{4mm}
}
\newcommand\numberthis{\addtocounter{equation}{1}\tag{\theequation}}
\usepackage[english]{babel}
%Includes "References" in the table of contents
\usepackage[nottoc]{tocbibind}
\begin{document}
	\homework{Homework x}{2023040163}{Zhao Han Hong}
	
	\section*{1 True or False Questions}
	\section*{Problem 1}

	Write your solution here.
	\section*{2 Multiple Choice Questions}
	\section*{Problem 1}

	Write your solution here.
	\section*{3 Short Answer Questions}
	\section*{Problem 1}

	Write your solution here.
	\section*{4 Technical Report}
	\begin{enumerate}[1.]
		\item Write your answer.
		\item Write your answer.
		\item ...
	\end{enumerate}
	\section*{Math Environments}
	
	Math expressions can be displayed inline, e.g., $a \times b = b \times a$, or
	with a displayed formula, e.g.,
	\[
	\sum_{i=1}^n i^2 = \frac{1}{6} n (n+1) (2n+1).
	\]
	You can also label a displayed formula
	\begin{equation} 
	\Pr[\text{No collision}] = \prod_{i=1}^k \Big( 1 - \frac{i-1}{n} \Big) \label{eq:example}
	\end{equation}
	and refer to it with \eqref{eq:example}. (Note: always use labels
	and \emph{never} write the numbers manually.)
	
	Use \textbf{align} environment for multi-line expressions:
	\begin{align*}
	x &= \ldots \numberthis \label{eq:align} \\
	&= \ldots \\
	&\le \ldots \numberthis
	\end{align*}
	Note that we use \textbf{align*} environment to prevent \LaTeX{}  
	from numbering every line, and use \textbf{\textbackslash numberthis} defined at the header to number some of the lines, and refer to them with \eqref{eq:align}.
	
	\newpage
	Use \textbf{enumerate} environment for a problem with several sub problems:
	
	
	\paragraph{Acknowlwedgements:} The author thanks XXX and YYY for discussion on Problem A.
	\bibliographystyle{unsrt}
    \bibliography{cite}
\end{document} 
